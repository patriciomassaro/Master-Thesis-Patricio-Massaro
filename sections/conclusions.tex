\section{Conclusions}

In this work, the feasibility of applying blind single-image super resolution algorithms to thermal remote sensing data from OroraTech’s FOREST-2 mission was studied. The study is centered on the following research questions:

\begin{itemize}
    \item What is the impact of the unknown FOREST-2 degradation model compared to the one commonly used for dataset generation?
    \item Is it possible to estimate the FOREST-2 degradation model using a data-driven approach?
    \item How can the degradation model be incorporated into training to improve SR results?
    \item  How can the SR results be assessed when paired data is scarce?
\end{itemize}

To have HR data, an automated process was developed for fetching, filtering, and processing data from third-party, higher-resolution missions. The conventional method for dataset pair generation of bicubic downsampling + white noise was proven inadequate in section \ref{sec:results}. The domain gap between the artificially degraded and real LR images was shown to be too wide, resulting in an SR model that is not able to generalize to FOREST-2 images and renders blurry results.

Blind super resolution techniques were explored as a solution in section \ref{sec:methodology}. The current lack of paired ECOSTRESS and FOREST-2 data leads to the use of implicit modeling for the data-driven degradation process estimation through a GAN architecture. A probabilistic degradation model was proposed for the task, mainly for three reasons. First, its well-constrained kernel and noise modules allow the training of an end-to-end pipeline of degradation and SR. Second, using deep learning to generate the kernel and noise facilitates the analysis of the degradation process and comprehension of its effects. Third, the stochastic nature of the proposed architecture allows sampling from the degradation distribution to generate a wide variety of HR-LR pairs from only one HR image, which is very useful for augmenting training datasets.

The degradation model adapted to FOREST-2 images and its effects were studied by comparing them to a baseline degradation in section \ref{sec:results}. The results showed that the adapted degradation model produces LR images that are consistently blurrier and have powerful noise that is usually strongly correlated with the content of the image. Although stronger degradation was introduced, the SR results for both pipelines are similar. 
% Thus, the SR model, despite starting from different LR images, can reconstruct the original image with comparable quality. 
This display of capacity suggests that the SR model is probably being under-utilized when using baseline degradations. However, an upper bound to the PSNR is observed, regardless of the LR input. The introduction of newer SISR methods or multi-spectral SR may be beneficial to overcome this limitation in reconstruction performance.

When comparing the results of the adapted SR model with the baseline on real FOREST-2 images, more detail and edges seem to be present at the cost of a small increment in the overall noise. The implemented framework of unpaired image comparison consisting of frequency domain, gradient and pixel-neighborhood correlation analysis also yielded positive outcomes. 
The images have more power in a wide range of frequencies when compared to bicubic upsampling, in the order of 6dB. The amplified frequencies match the ones lost during the degradation process, implying that the adapted SR model is able to recover part of the signal. The observations from the gradient magnitude and pixel-neighborhood correlation analysis corroborate the results. Moreover, the model was also tested in a single paired scene from ECOSTRESS and FOREST-2, revealing it has higher performance and can highlight subtle features such as small islands that were almost undetectable using traditional upsampling methods. Nevertheless, the limited number of available images does not allow for drawing definitive conclusions.

The sensitivity to differences between training data and arbitrary inputs is a very relevant result, as it shows the limitations of the implicit modeling approaches. While they help to bridge the domain gap, they are not able to generalize to an arbitrary input that is outside of the target domain.
When applying the adapted SR model to bicubic downsampled images, the resulting image quality proved to be unusable.
In this case, the direction of the gap is inverted. The estimated degradation kernel is more complex than the actual one, resulting in SR over-amplifying frequencies. This also highlights the difficulty of hand-picking the degradation model in classical dataset generation techniques, as it is easy to be overly optimistic or pessimistic when choosing the amount of degradation.
The sensitivity to domain shifts was also analyzed using non-referenced image quality assessment metrics, where the domain-adapted SR model had a significantly better score only when the input was real FOREST-2 images. 

Overall, the combination of probabilistic degradation modeling and SR yielded promising results. It is a flexible approach in the sense that it only requires two sufficiently large datasets that do not need to be paired. 
The drawbacks of implicit modeling are not as relevant in this application compared to other tasks because the conditions of the missions (and their degradation models) are almost static. Unlike other applications such as smartphone images, where the number of possible cameras and sensors is practically infinite, the number of sensors in a satellite remains the same, and only the change of conditions due to the passing of time should be considered. This means that the degradation model can be trained on a particular domain, and the LR input that will be used in the future will probably come from the same distribution. The end-to-end nature of this training framework allows to have multiple SR models, one for each wanted target domain, with low difficulty.

% The implemented framework leverages implicit modeling to estimate a degradation model and produce LR images that can be analyzed to study its impact on the SR process. This information is also incorporated into the training process to produce better results.
% The development of methods for assessing SR performance without paired HR-LR images is also an important contribution due to the scarce nature of this type of data. While they may not be sufficient to quantify the performance of the SR model, they are a good indicator of the quality of the results.

\subsection{Future Work}

This work has laid the groundwork for doing degradation-aware super resolution of FOREST-2 images without the need for paired data. While the results are promising, several assumptions can be challenged, and many avenues remain unexplored. The following points outline promising directions for future research:

\begin{itemize}

    \item The HR dataset obtained from ECOSTRESS is based on the similarities in the spectral domain between the two missions. While their characteristics are very similar, they are not the same, and the implications of this mismatch should be further studied.
    
    \item Despite using unpaired data for training, the lack of a paired HR-LR dataset still poses a challenge. One single scene does not allow for drawing definitive conclusions on the performance of the model. Moreover, the availability of a paired dataset composed of more images would allow for better training decisions, leveraging techniques like early stopping for model selection. 
    
    \item The probabilistic degradation modeling assumes independence between the noise and kernel components. While it is a reasonable assumption, it may not hold in all cases.
    
    \item The addition of a discriminator that distinguishes between the HR image and the super-resolved  LR images coming from the target domain may improve the quality of the results even further. It may do so by aligning their distributions without the need for them to be paired.
    
    \item Due to the sensibility to any domain shift, investigating pre-processing methods for detecting inputs that are out-of-distribution with respect to the training data before using the SR models may prove helpful in avoiding undesirable results.
    
    \item The introduction of novel SISR architectures, or the incorporation of information from other spectral bands to use MSSR as in \cite{myself2023}, may be beneficial to overcome the limitations of the SR model. If multi-image data from FOREST-2 is available, MISR is also a very promising alternative.
    
    \item In the future, OroraTech will launch a constellation of cubesats that will have identical hardware. Still, the degradation model may be slightly different due to manufacturing tolerances and performance degradation over time. When more data is available, it will be necessary to investigate whether a general model for all FOREST data products is possible or if each cubesat needs its own model.
    
    \item While the NIQE and BRISQUE metrics help assess the quality of SR without a reference, their corresponding models are trained on natural images. The development of NIQE and BRISQUE metrics trained on remote sensing data may be more suitable for the task.
    \item Because of how the frequency domain analysis is done, the value of a particular spatial frequency is the average over every direction. This approach could be further developed to focus on specific directions.
    
\end{itemize}