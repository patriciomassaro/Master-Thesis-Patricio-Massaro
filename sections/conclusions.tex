\section{Conclusions and future work}

\begin{itemize}
    \item Flexible approach
    \item Models don't need to be very general, they just have to adapt between two very distinctive domains.
    \item Conditions of each missions are almost static
    \item Severe lack of paired data but abundance of unpaired data
    \item Just give me two datasets and the pipeline will find the way.
\end{itemize}


degradation model assumes complete independence between the noise and kernel components.
It is a reasonable assumption but it may not be true in all cases.

The domain gap is not only being very optimistic when building the dataset. 
You can also be very pesimistic and lead to catastrophic results.
Highlight on the difficulty of hand-picking the amount of degradation and the complexity of degradation modeling.
Domain adaption seems to be very suitable due to: 




Training of non-referenced image quality assesment for remote sensing
MSSR
More paired datasets would allow a more robust training of the model, because we could try to maximize the PSNR in those cases instead of how we do it now.
