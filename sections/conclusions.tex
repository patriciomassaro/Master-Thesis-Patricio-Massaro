\section{Conclusions and future work}

In this thesis, the feasibility of applying blind single-image super resolution algorithms to thermal remote sensing data in real world environments was studied with satisfactory results. The framework used is very flexible, requiring only two unpaired datasets with enough data. Additionally, one of the contributions of this work is the development of methods that may help in the assessment of performance without the need for paired data. While they are not sufficient to quantify the performance of the SR model, they are a good indicator of the quality of the results.

The work started , by introducing the fundamentals of remote sensing, the importance of thermal infrared data for land surface temperature estimation, and the dimensions of quality that they present.
Applications of LST products like wildfire management and urban heat studies require high spatial and temporal resolutions, which are not available in the current options. 
Because of different variables like payload capacity of the missions, sensor technology and the cost of the missions, there is a trade-off betweem spatial and temporal resolutions that is not possible to circumvent at the present day.
While Ororatech has a plan to improve the temporal resolution using swarms of small satellites like FOREST-2, the spatial resolution remains a challenging problem.

Super resolution techniques offer a promising solution, being SISR the most straightforward of them. As a supervised problem, they require a paired dataset of HR-LR images. This is not possible for FOREST-2, because HR images are not available.
To circumvent this issue, data from external missions with better resolution like ECOSTRESS was used to generate a syntheticly generated HR FOREST-2 dataset. ECOSTRESS is an ideal candidate for the task because of its spectral and spatial characteristics that match very well with FOREST-2.
An automated process for fetching the data from the NASA API,and process it was developed.

However, it is still not possible to obtain paired data, as the scenes from ECOSTRESS and FOREST-2 are different. Most of the literature in super resolution rely on downscaling the HR images using a gaussian kernel and adding white noise to generate the LR pairs required for training. 
In the results section, it was proved that this approach is not up to the task. The domain gap between the synthetically degraded and real images is too big , resulting in an SR model that is not able to generalize to FOREST-2 images, producing blurry images.

Blind super resolution techniques were explored as a solution. The availability of FOREST-2 data drived the decision of using a implicit modelling of the degradation process, through a GAN architecture for the degradation. A probabilistic degradation model was proposed for the task, as its well-constrained kernel and noise modules allow the training of an end-to-end pipeline of degradation + SR. Additionally, sampling from the degradation distribution allows to generate a wide variety of HR-LR pairs from only one HR image, which is very useful for the training of the SR model.

The degradation model adapted to FOREST-2 images and it's effects were studied by comparing them to a baseline gaussian + noise degradation. The results showed that this adapted model produces LR images that are blurrier, and consistently with more noise than the baseline, implying that FOREST-2 degradation model is more complex.
Its kernel module is not gaussian, and the noise has a big dependence on the content of the image, implying that the noise is not white. 
The LR images generated by the adapted degradation model are more challenging for SR than the baseline, as proven when applying bicubic upsampling to them and comparing them with the HR.  
The frequency domain analysis showed similar conclusions, where the adapted degradation model produced images a more narrow frequency response than the baseline.

Surprisingly enough, the result of super resolving LR images coming from the adapted degradation model and the baseline yielded very similar results. This means that the SR model, despite starting from very different LR images, was able to reconstruct the HR image with similar quality. This displays the power of the SR model, and the fact that it is being under-utilized when using bicubic downsampled LR images. Despite this, a "hard-limit" for the SR is observed and there seems to be a upper bound to the PSNR that the model is being able to obtain. The introduction of newer SISR methods or multi-spectral SR may be beneficial to overcome this limitation.

When comparing the results of the adapted SR model with the baseline on real FOREST-2 images, a lot more of detail and edges seem to be present, at a cost of a small increment in the overall noise. 
The images have stronger components in a range of frequencies, compared with bicubic upsampling, in the order of 6dB. Additionally, the frequencies match the ones lost during the degradation process, implying that the adapted SR model is able to recover the lost frequencies during the degradation process.
This phenomena can also be observed by analyzing the gradient magnitude of the images, where the adapted SR model produces images that tend to have bigger gradients, implying sharper edges. 
While the model was trained on thermal infrared data, it was proven robust to be used directly on LST products. This allows to improved the spatial resolution of LST images that were already processed from TIR data, without the need of reprocessing them.

However, the model is still very sensitive to the domain gap. 
When applying the adapted SR model to bicubic downsampled images, the results are catastrophic.
The edges of the images seem to be sharper but several artifacts appear, and the frequencies usually related to noise are dangerously amplified. 
In this case, the common situation is reversed, the estimated degradation kernel is more complex than the actual one, resulting in "over-amplification". This also highlights the difficulty of hand-picking the degradation model in classical dataset generation techniques, as it is very easy to be very optimistic or very pessimistic when choosing the amount of degradation.
The phenomena was also analyzed using non-referenced image quality assessment metrics, where the adapted SR model has significantly better score only when the input are real FOREST-2 images. Additionally, the adapted SR model was the only model that got worse scores when the input was bicubic downsampled images instead of real FOREST-2 images.
The sensitivity to differences between training and arbitrary inputs is a very relevant result, as it shows the limtations of the implicit modelling approaches. While they clearly help to bridge the domain gap, they are not able to generalize to an arbitrary input that is outside of the target domain.


Overall, the combination of probabilistic degradation modelling and SR yielded very promising results. It is a very flexible approach, in the sense that it only requires two sufficiently large datasets that  don't need to be paired. The latter is a very important advantage, as it is very difficult to obtain paired data in remote sensing, but unpaired data is abundant. The mentioned drawbacks of implicit modelling are not as relevant here, compared to other tasks, because the conditions of the missions (and their degradation models) are almost static. Unlike other domains like natural images, where the amount of possible cameras and sensors are almost infinite, the number of sensors and in a satellite remains the same and only the change of conditions because of time should be taken into account. This means that the degradation model can be trained on a very specific domain, and the LR input that will be used in the future will probably come from the same distribution.
Fortunately, the low difficulty of training models with this approach allows to have multiple SR models, one for each wanted target domain, with low effort.

Despite using unpaired data for training, the lack of a paired HR-LR dataset still poses a challenge. While the results are promising, it is not possible to quantify how good the adapted SR model in super resolving real FOREST-2 images compared to other methods due to the lack of a ground truth HR image. 
This also provokes a limitation in model training. Right now, the best model during training is chosen as the one that maximizes the PSNR on the source domain of the validation set (That means, taking the HR image, degrading it to LR, super resolving it and comparing to the HR). While this is not a bad criteria at all, it would be better to pick the one that maximizes the PSNR on the target domain ( directly upsampling the real FOREST-2 images), as it will be the relevant metric in a production environment. In order to do this, a paired dataset of at least 100 images of real FOREST-2 images paired with their HR counterparts is needed.

While the super resolution model is already being used in LST calculation pipelines of OroraTech, there is still room for improvement. Possible directions of future work will be discussed further:

\begin{itemize}
    \item The probabilistic degradation modelling assumes complete independence between the noise and kernel components. While it is a reasonable assumption, it may not be true in all cases.
    \item As explored in the implicit modelling section. The addition of a discriminator that distinguishes between the HR image and the super resolved real FOREST-2 images may improve the quality of the results even further by aligning their distributions without the need of them being paired.
    \item In the future, OroraTech will launch a swarm of small satellites called FOREST-2. They will have similar properties, but the degradation model may be different. When the data is available, investigating wheter a general model for all FOREST data products is possible or if each mission needs its own model will be interesting.
    \item While NIQE and BRISQUE may be helpful to assess the quality of the Super resolution without a reference, their corresponding models are trained on natural images. The development of NIQE/BRISQUE metrics trained on remote sensing data may be more suitable for the task.
    \item The introduction of newer SISR architectures, or the incorporation of information from other spectral bands to do MSSR as in \cite{myself2023} may be beneficial to overcome the "hard-limit" of the SR model. MISR is also very promising, but the difficulty to obtain multi-image data from FOREST-2 is a challenging limiation for its application.
    \item Because of how the frequency domain analysis is done, the value for one normalized spatial frequency is the average on every direction. This approach could be further developed to focus on specific directions, as the real degradation model is probably not isotropic.
\end{itemize}